% Options for packages loaded elsewhere
\PassOptionsToPackage{unicode}{hyperref}
\PassOptionsToPackage{hyphens}{url}
\documentclass[
]{book}
\usepackage{xcolor}
\usepackage{amsmath,amssymb}
\setcounter{secnumdepth}{5}
\usepackage{iftex}
\ifPDFTeX
  \usepackage[T1]{fontenc}
  \usepackage[utf8]{inputenc}
  \usepackage{textcomp} % provide euro and other symbols
\else % if luatex or xetex
  \usepackage{unicode-math} % this also loads fontspec
  \defaultfontfeatures{Scale=MatchLowercase}
  \defaultfontfeatures[\rmfamily]{Ligatures=TeX,Scale=1}
\fi
\usepackage{lmodern}
\ifPDFTeX\else
  % xetex/luatex font selection
\fi
% Use upquote if available, for straight quotes in verbatim environments
\IfFileExists{upquote.sty}{\usepackage{upquote}}{}
\IfFileExists{microtype.sty}{% use microtype if available
  \usepackage[]{microtype}
  \UseMicrotypeSet[protrusion]{basicmath} % disable protrusion for tt fonts
}{}
\makeatletter
\@ifundefined{KOMAClassName}{% if non-KOMA class
  \IfFileExists{parskip.sty}{%
    \usepackage{parskip}
  }{% else
    \setlength{\parindent}{0pt}
    \setlength{\parskip}{6pt plus 2pt minus 1pt}}
}{% if KOMA class
  \KOMAoptions{parskip=half}}
\makeatother
\usepackage{color}
\usepackage{fancyvrb}
\newcommand{\VerbBar}{|}
\newcommand{\VERB}{\Verb[commandchars=\\\{\}]}
\DefineVerbatimEnvironment{Highlighting}{Verbatim}{commandchars=\\\{\}}
% Add ',fontsize=\small' for more characters per line
\usepackage{framed}
\definecolor{shadecolor}{RGB}{248,248,248}
\newenvironment{Shaded}{\begin{snugshade}}{\end{snugshade}}
\newcommand{\AlertTok}[1]{\textcolor[rgb]{0.94,0.16,0.16}{#1}}
\newcommand{\AnnotationTok}[1]{\textcolor[rgb]{0.56,0.35,0.01}{\textbf{\textit{#1}}}}
\newcommand{\AttributeTok}[1]{\textcolor[rgb]{0.13,0.29,0.53}{#1}}
\newcommand{\BaseNTok}[1]{\textcolor[rgb]{0.00,0.00,0.81}{#1}}
\newcommand{\BuiltInTok}[1]{#1}
\newcommand{\CharTok}[1]{\textcolor[rgb]{0.31,0.60,0.02}{#1}}
\newcommand{\CommentTok}[1]{\textcolor[rgb]{0.56,0.35,0.01}{\textit{#1}}}
\newcommand{\CommentVarTok}[1]{\textcolor[rgb]{0.56,0.35,0.01}{\textbf{\textit{#1}}}}
\newcommand{\ConstantTok}[1]{\textcolor[rgb]{0.56,0.35,0.01}{#1}}
\newcommand{\ControlFlowTok}[1]{\textcolor[rgb]{0.13,0.29,0.53}{\textbf{#1}}}
\newcommand{\DataTypeTok}[1]{\textcolor[rgb]{0.13,0.29,0.53}{#1}}
\newcommand{\DecValTok}[1]{\textcolor[rgb]{0.00,0.00,0.81}{#1}}
\newcommand{\DocumentationTok}[1]{\textcolor[rgb]{0.56,0.35,0.01}{\textbf{\textit{#1}}}}
\newcommand{\ErrorTok}[1]{\textcolor[rgb]{0.64,0.00,0.00}{\textbf{#1}}}
\newcommand{\ExtensionTok}[1]{#1}
\newcommand{\FloatTok}[1]{\textcolor[rgb]{0.00,0.00,0.81}{#1}}
\newcommand{\FunctionTok}[1]{\textcolor[rgb]{0.13,0.29,0.53}{\textbf{#1}}}
\newcommand{\ImportTok}[1]{#1}
\newcommand{\InformationTok}[1]{\textcolor[rgb]{0.56,0.35,0.01}{\textbf{\textit{#1}}}}
\newcommand{\KeywordTok}[1]{\textcolor[rgb]{0.13,0.29,0.53}{\textbf{#1}}}
\newcommand{\NormalTok}[1]{#1}
\newcommand{\OperatorTok}[1]{\textcolor[rgb]{0.81,0.36,0.00}{\textbf{#1}}}
\newcommand{\OtherTok}[1]{\textcolor[rgb]{0.56,0.35,0.01}{#1}}
\newcommand{\PreprocessorTok}[1]{\textcolor[rgb]{0.56,0.35,0.01}{\textit{#1}}}
\newcommand{\RegionMarkerTok}[1]{#1}
\newcommand{\SpecialCharTok}[1]{\textcolor[rgb]{0.81,0.36,0.00}{\textbf{#1}}}
\newcommand{\SpecialStringTok}[1]{\textcolor[rgb]{0.31,0.60,0.02}{#1}}
\newcommand{\StringTok}[1]{\textcolor[rgb]{0.31,0.60,0.02}{#1}}
\newcommand{\VariableTok}[1]{\textcolor[rgb]{0.00,0.00,0.00}{#1}}
\newcommand{\VerbatimStringTok}[1]{\textcolor[rgb]{0.31,0.60,0.02}{#1}}
\newcommand{\WarningTok}[1]{\textcolor[rgb]{0.56,0.35,0.01}{\textbf{\textit{#1}}}}
\usepackage{longtable,booktabs,array}
\usepackage{calc} % for calculating minipage widths
% Correct order of tables after \paragraph or \subparagraph
\usepackage{etoolbox}
\makeatletter
\patchcmd\longtable{\par}{\if@noskipsec\mbox{}\fi\par}{}{}
\makeatother
% Allow footnotes in longtable head/foot
\IfFileExists{footnotehyper.sty}{\usepackage{footnotehyper}}{\usepackage{footnote}}
\makesavenoteenv{longtable}
\usepackage{graphicx}
\makeatletter
\newsavebox\pandoc@box
\newcommand*\pandocbounded[1]{% scales image to fit in text height/width
  \sbox\pandoc@box{#1}%
  \Gscale@div\@tempa{\textheight}{\dimexpr\ht\pandoc@box+\dp\pandoc@box\relax}%
  \Gscale@div\@tempb{\linewidth}{\wd\pandoc@box}%
  \ifdim\@tempb\p@<\@tempa\p@\let\@tempa\@tempb\fi% select the smaller of both
  \ifdim\@tempa\p@<\p@\scalebox{\@tempa}{\usebox\pandoc@box}%
  \else\usebox{\pandoc@box}%
  \fi%
}
% Set default figure placement to htbp
\def\fps@figure{htbp}
\makeatother
\setlength{\emergencystretch}{3em} % prevent overfull lines
\providecommand{\tightlist}{%
  \setlength{\itemsep}{0pt}\setlength{\parskip}{0pt}}
\usepackage[]{natbib}
\bibliographystyle{apalike}
\usepackage{booktabs}
\usepackage{amsthm}
\makeatletter
\def\thm@space@setup{%
  \thm@preskip=8pt plus 2pt minus 4pt
  \thm@postskip=\thm@preskip
}
\makeatother
\usepackage{bookmark}
\IfFileExists{xurl.sty}{\usepackage{xurl}}{} % add URL line breaks if available
\urlstyle{same}
\hypersetup{
  pdftitle={A Minimal Book Example},
  pdfauthor={Yihui Xie},
  hidelinks,
  pdfcreator={LaTeX via pandoc}}

\title{A Minimal Book Example}
\author{Yihui Xie}
\date{2025-01-12}

\begin{document}
\maketitle

{
\setcounter{tocdepth}{1}
\tableofcontents
}
\chapter{Prerequisites}\label{prerequisites}

This is a \emph{sample} book written in \textbf{Markdown}. You can use anything that Pandoc's Markdown supports, e.g., a math equation \(a^2 + b^2 = c^2\).

The \textbf{bookdown} package can be installed from CRAN or Github:

\begin{Shaded}
\begin{Highlighting}[]
\FunctionTok{install.packages}\NormalTok{(}\StringTok{"bookdown"}\NormalTok{)}
\CommentTok{\# or the development version}
\CommentTok{\# devtools::install\_github("rstudio/bookdown")}
\end{Highlighting}
\end{Shaded}

Remember each Rmd file contains one and only one chapter, and a chapter is defined by the first-level heading \texttt{\#}.

To compile this example to PDF, you need XeLaTeX. You are recommended to install TinyTeX (which includes XeLaTeX): \url{https://yihui.org/tinytex/}.

\begin{center}\rule{0.5\linewidth}{0.5pt}\end{center}

\chapter{拓扑学}\label{intro}

\section{如何判定是否为度量空间?}\label{ux5982ux4f55ux5224ux5b9aux662fux5426ux4e3aux5ea6ux91cfux7a7aux95f4}

\(X\)是一个非空集合,\(\rho: X\times X\rightarrow \mathbb{R}\)是一个函数.对于任意的\(x,y,z\in X\), 满足:

\[
\begin{aligned}
1. &\text{正定性,} \rho(x,y) \ge 0, \text{当} x = y \text{时}, \rho(x,y) = 0 \\
2. &\text{对称性,} \rho(x,y) = \rho(y,x) \\
3. &\text{三角不等式,} \rho(x,y) \le \rho(x,z) + \rho(y,z)
\end{aligned}
\]

\(\rho是X\)上的一个度量,\textbf{或者距离},这里用距离就很好理解了。\((X,\rho)或X\)为一个度量空间

\section{欧式度量}\label{ux6b27ux5f0fux5ea6ux91cf}

如果我们把以上的度量空间从2维度推广到n维,那么就可以得到欧式空间

\[
\rho(x,y) = (\sum_{k=1}^{n}(x_k-y_k)^{\frac{1}{2}})
\]

\section{球形邻域}\label{ux7403ux5f62ux90bbux57df}

\section{开集}\label{ux5f00ux96c6}

设\((X,\rho)\)是一个度量空间,\(A\subset X\).\(\forall x\in A\),那么都存在一个\(x的球形邻域B(x,\epsilon)\),使得: \[x\in B(x,\epsilon)\subset A\] 成立,那么\(A\)就是\((X,\rho)\)的开集

任意球形邻域都是开集

度量空间里的一个集合是开集\(\Leftrightarrow\)这个集合可以表示为若干球形邻域的并

\section{开集的基本性质}\label{ux5f00ux96c6ux7684ux57faux672cux6027ux8d28}

设\(\mathcal{T}\)是\((X,\rho)\)中所有的开集构成的集合,\(\mathcal{T}\)满足如下性质:

\begin{enumerate}
\def\labelenumi{\arabic{enumi}.}
\item
  \(\emptyset,X\in \mathcal{T}\)
\item
  有限多个\(\mathcal{T}\)的交依然是\(\mathcal{T}\)的成员
\item
  任意多个\(\mathcal{T}\)的并依然是\(\mathcal{T}\)的成员
\end{enumerate}

\section{拓扑}\label{ux62d3ux6251}

\(设X是一个非空集合,如果\mathcal{T}\subset 2^X\)满足:

\[1. \emptyset,X\in \mathcal{T}\\
2.  有限多个\mathcal{T}交依然是\mathcal{T}的成员\\
3.  任意多个\mathcal{T}的并依然是\mathcal{T}的成员\\
那么\mathcal{T}是X上的拓扑,(X,\mathcal{T})是一个拓扑空间,\mathcal{T}中的成员都是(X,\mathcal{T})的开集
\]

平凡拓扑是最小的拓扑只包含\(\emptyset\)和\(X\),离散拓扑是最大的拓扑,\(\mathcal{T}=2^X\)

\section{闭包=稠密}\label{ux95edux5305ux7a20ux5bc6}

设\((X,\mathcal{T})是拓扑空间,x\in X,A\subset X,如果对于x的任意邻域U,都有\)

\(U\cap (A-{x})\neq \emptyset\)

那么\(x是一个聚点,A所有聚点的集合叫做A的导集,记作A',那么A的闭包\overline{A}=A\cup A'\)

\section{内部边界}\label{ux5185ux90e8ux8fb9ux754c}

\(拓扑空间(X,\mathcal{T}),A\subset X,x\in A,如果A是邻域,那么x是内点,所有内点集合就是内部\)

\(拓扑空间(X,\mathcal{T}),A\subset X,x\in,如果对于任意x的邻域U,都有\)

\(U\cap A\neq \emptyset而且U\cap A^C\neq \emptyset\)

\(那么x就是A的边界点,边界点集合就是A的边界\)

\section{拓扑基、子基}\label{ux62d3ux6251ux57faux5b50ux57fa}

\(拓扑空间(X,\mathcal{T}),\mathcal{B} \subset \mathcal{T},如果空间里每个开集都能表示为\mathcal{B}里面若干成员的并,那么\mathcal{B}是\mathcal{T}的一个基(拓扑基)\)

\(拓扑空间(X,\mathcal{T}),\mathcal{S}\subset \mathcal{T},如果\mathcal{S}中元的有限交全体构成\mathcal{T}的拓扑基,那么\mathcal{S}是\mathcal{T}的拓扑子基\)

\section{两个例子}\label{ux4e24ux4e2aux4f8bux5b50}

R上的开区间生成的拓扑 余有限拓扑

\section{子空间}\label{ux5b50ux7a7aux95f4}

\(设\mathcal{T}是一个集族,Y是一个集合,那么\)

\(\mathcal{T}|Y = \{A\cap Y|A\in \mathcal{T}\}是集族\mathcal{T}在集合Y上的限制\)

\(拓扑空间(X,\mathcal{T}),\emptyset \neq Y \subset X,那么\mathcal{T}_Y = \mathcal{T}|Y是Y上的拓扑\)

\section{乘积空间}\label{ux4e58ux79efux7a7aux95f4}

\[
拓扑空间(X_1,\mathcal{T}_1)(X_1,\mathcal{T_2}),\mathcal{B} = \{U_1\times U_2|U_i\in \mathcal{T}_i\},\\
那么\overline{ \mathcal{B}} = \{U\subset X_1\times X_2|U是\mathcal{B}中的若干成员\}为X_1\times X_2上由\mathcal{B}生成的乘积拓扑,\\
(X_1\times X_2,\overline{\mathcal{B}})就是拓扑空间(X_1,\mathcal{T}_1)(X_1,\mathcal{T_2})的乘积空间
\]

\section{商空间}\label{ux5546ux7a7aux95f4}

\[
拓扑空间(X,\mathcal{T}),Y是一个集合,f:X\rightarrow Y是一个满射,\\
那么Y的子集族\widetilde{\mathcal{T}}=\{U\subset Y|f^{-1}(U)\in \mathcal{T}\}\\
是Y的拓扑,这个拓扑是Y上的一个商拓扑
\]

\section{紧性}\label{ux7d27ux6027}

\section{连通性}\label{ux8fdeux901aux6027}

\section{hausdaff性质}\label{hausdaffux6027ux8d28}

\chapter{Literature}\label{literature}

Here is a review of existing methods.

\chapter{Methods}\label{methods}

We describe our methods in this chapter.

Math can be added in body using usual syntax like this

\section{math example}\label{math-example}

\(p\) is unknown but expected to be around 1/3. Standard error will be approximated

\[
SE = \sqrt{\frac{p(1-p)}{n}} \approx \sqrt{\frac{1/3 (1 - 1/3)} {300}} = 0.027
\]

You can also use math in footnotes like this\footnote{where we mention \(p = \frac{a}{b}\)}.

We will approximate standard error to 0.027\footnote{\(p\) is unknown but expected to be around 1/3. Standard error will be approximated

  \[
  SE = \sqrt{\frac{p(1-p)}{n}} \approx \sqrt{\frac{1/3 (1 - 1/3)} {300}} = 0.027
  \]}

\chapter{Applications}\label{applications}

Some \emph{significant} applications are demonstrated in this chapter.

\section{Example one}\label{example-one}

\section{Example two}\label{example-two}

\chapter{Final Words}\label{final-words}

We have finished a nice book.

\bibliography{book.bib,packages.bib}

\end{document}
